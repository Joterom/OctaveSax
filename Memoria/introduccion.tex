Un pedal de efectos es un dispositivo que se conecta entre un instrumento (normalmente electrófono) y su amplificador, modificando la señal de entrada y sus características fundamentales como pueden ser timbre, tono y volumen. En mi caso particular he optado por un pedal para saxofón, puesto que es el instrumento que yo toco, aunque resulta igual de válido para cualquier otro intrumento cuyo sonido se pueda captar con un micrófono convencional. Generalmente, no es habitual el uso de pedales de efectos en instrumentos de viento debido a que se suele buscar un sonido limpio y fiel. No obstante, en multitud de producciones se pueden apreciar efectos añadidos en postproducción ya sea digital o analógicamente. Algunos ejemplos son el \emph{reverb, chorus u octavador}. Es este último el efecto que se implementará en este proyecto, con la salvedad de que funcionará en tiempo real. 

Se ha decidido el combinar este efecto con la posibilidad de implementarlo en formato de pedal el cual se ha hecho muy popular desde su aparición, debido a que los intérpretes pueden activarlo con el pie pudiendo mantaner las manos en el instrumento. En el caso del saxofón, esto no suele ser un problema, ya que los músicos no precisan de estar tocando de forma continuada, si no que suele haber un tiempo para descansar suficientemente largo como para operar el dispositivo que se este utilizando. A pesar de ello, resulta conveniente este formato por analogía con otros instrumentos.

Este proyecto abarca todo el proceso desde la idea inicial de diseño hasta el montaje del prototipo final, por tanto, las especificaciones de funcionamiento que se han utilizado pretenden facilitar un uso profesional del prototipo, de forma que sea compatible con los estándares establecidos en el contexto musical e ingenieril al mismo tiempo.

En primer lugar se utiliza un transductor para adquirir la señal, en este caso, un micrófono convencional. Una vez que el estímulo externo es transformado en pulsos eléctricos, atravesará un etapa de entrada analógica que pre-amplifica la señal y la adecua a la entrada de la FPGA.

Para el prototipo se ha utilizado la placa proporcionada por el departamento: Nexys A7. Esta placa monta una FPGA \emph{Xilinx XC7A100T-1CSG324C} junto con varios switches, botones, leds y displays de 7 segmentos, que harán más fácil el manejo del prototipo. Esta placa tiene un micrófono integrado, pero es de tan baja calidad que se opta por diseñar la etapa de entrada, analógica completamente, y conectarla con uno de los puertos del \emph{Pmod i2s2}, también de Digilent y proporcionado por el departamento. Este módulo contiene ADC, DAC y los conectores de mini-jack estándar en formato de audio, que servirán para gestionar la señal de entrada y la de salida.

Se implementará un algoritmo que se encargará de llevar a cabo la octavación de la señal de entrada y de proporcionarla en la salida. Este algoritmo utiliza una aproximación de \emph{Phase Vocoder} muy común en el tratamiento de señales de audio realizando una transformación al dominio de la frecuencia mediante FFT. Durante todo el proceso se priorizará el criterio de la baja latencia, dado que si no, resulta imposible la interpretación para el músico. El algoritmo se describirá en profundidad en su capítulo correspondiente.

\section{Objetivos}
Para la realización de este proyecto de forma satisfactoria se ha establecido la consecución de una serie de objetivos que son los siguientes:
\begin{itemize}
\item Diseño de un sistema completo, a partir de una problemática definida previamente mediante el estudio del instrumento y las señales.
\item Implementación sobre la FPGA proporcionada.
\item Construcción de un prototipo para estudiar la problemática desde la práctica.
\item Afianzar, debido a todo esto, conocimientos adquiridos durante el Grado en diversos ámbitos como el procesamiento de señal y en concreto en la especialidad de Sistemas Electrónicos como la programación hardware, el montaje de un circuito analógico y la correcta comunicación entre todos los módulos.
\end{itemize}

\section{Metodología}
\begin{itemize}
\item En primer lugar se seleccionará el efecto que se quieren implementar, dando prioridad al octavador.
\item Como segundo paso, se estudiará la literatura existente y se probarán distintas soluciones algorítmicas empleando MATLAB.
\item Estudio y elección de la interfaces de entrada y salida. Selección de micrófono, ADC y DAC.
\item Desarrollo y verificación en VHDL empleando la herramienta VIVADO de Xillinx.
\item Montaje de un prototipo empleando la placa Nexys A7 de Digilent, el micrófono seleccionado anteriormente y el resto de dispositivos que fueran necesarios.
\item Test y depuración.
\end{itemize}