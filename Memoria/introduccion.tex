Un pedal de efectos es un dispositivo que se conecta entre un instrumento (normalmente electrófono) y su amplificador encargándose de modificar la señal de entrada y sus características fundamentales como pueden ser timbre, tono y volumen. En este proyecto se pretende estudiar las posibilidades de diseño e implementación de uno de estos pedales, enfocándolo a un uso con instrumentos de viento, en concreto el saxofón puesto que es el instrumento que yo toco. Generalmente, no es habitual el uso de pedales de efectos en instrumentos de viento, ya que en general se busca mantener el sonido lo más fiel posible al producido por el instrumento. No obstante, en multitud de producciones se pueden apreciar efectos añadidos en postproducción ya sea digital o analógicamente. Algunos ejemplos son el \emph{reverb} o el \emph{chorus}. Sin embargo, aqui se pretende implementar un efecto de octavador, el cual se describirá posteriormente.

He decidido el combinar este efecto con la implementación en formato de pedal. Este formato se ha hecho muy popular desde su aparición, debido a que los intérpretes pueden activarlo con el pie pudiendo mantaner las manos en el instrumento. Aunque los intérpretes de instrumentos de viento no están acostumbrados al uso de pedales, se mantiene la idea del pedal por analogía con otros instrumentos.

Este proyecto abarca todo el proceso desde la idea inicial, diseño y montaje del prototipo final, por tanto, las especificaciones de funcionamiento que se han utilizado pretenden facilitar un uso profesional del prototipo, de forma que sea compatible con los estándares establecidos en el contexto musical e ingenieril al mismo tiempo.

El flujo de datos será el siguiente. En primer lugar se utiliza un transductor para adquirir la señal, en este caso, un micrófono convencional. Una vez que el estímulo externo es transformado en pulsos eléctricos, atravesará un etapa de entrada analógica que pre-amplifica la señal y la adecúa a la entrada de la FPGA.

Para el prototipo se ha utilizado la placa proporcionada por el departamento: Nexys A7. Esta placa monta una FPGA \emph{Xilinx XC7A100T-1CSG324C} junto con varios switches, botones, leds y displays de 7 segmentos, que harán más fácil el manejo de la misma. Esta placa tiene un micrófono integrado, pero es de tan baja calidad que se opta por diseñar la etapa de entrada, analógica completamente, y conectarla con uno de los puertos del \emph{Pmod i2s2}, también de Digilent y proporcionado por el departamento. Este módulo contiene ADC, DAC y los conectores de mini-jack estándar en formato de audio, que servirán para gestionar la señal de entrada y la de salida.

Se implementará un algoritmo que se encargará de llevar a cabo la octavación de la señal de entrada y de proporcionarla en la salida. Este algoritmo utiliza una aproximación de \emph{Phase Vocoder} muy común en el tratamiento de señales de audio realizando una transformación al dominio de la frecuencia mediante FFT. Durante todo el proceso se priorizará el criterio de la \emph{baja latencia}, dado que si no, resulta imposible operar en tiempo real. El algoritmo se describirá en profundidad en su capítulo correspondiente.

Como se puede deducir el tema elegido se ha acordado directamente con el tutor sin que estuviera previamente propuesto por él o por el departamento.

\section{Objetivos}
Para la realización de este proyecto de forma satisfactoria se ha establecido la consecución de una serie de objetivos que son los siguientes:
\begin{itemize}
\item Diseño de un sistema completo, a partir de una problemática definida previamente mediante el estudio del instrumento y las señales.
\item Implementación sobre la FPGA proporcionada.
\item Construcción de un prototipo para estudiar la problemática desde la práctica.
\item Afianzar, debido a todo esto, conocimientos adquiridos durante el Grado en diversos ámbitos como el procesamiento de señal y en concreto en la especialidad de Sistemas Electrónicos como la programación hardware, el montaje de un circuito analógico y la correcta comunicación entre todos los módulos.
\end{itemize}

\section{Metodología}
\begin{itemize}
\item En primer lugar se seleccionará el efecto que se quieren implementar, dando prioridad al octavador.
\item Como segundo paso, se estudiará la literatura existente y se probarán distintas soluciones algorítmicas empleando MATLAB.
\item Estudio y elección de la interfaces de entrada y salida. Selección de micrófono, ADC y DAC.
\item Desarrollo y verificación en VHDL empleando la herramienta VIVADO de Xillinx.
\item Montaje de un prototipo empleando la placa Nexys A7 de Digilent, el micrófono seleccionado anteriormente y el resto de dispositivos que fueran necesarios.
\item Test y depuración.
\end{itemize}
\newpage
\section{Resultados}
Transcurrido el tiempo de desarrollo del proyecto, que ha sido de un año natural, se han alcanzado varias metas aunque no se ha podido ver terminada una versión funcional del prototipo. 
\begin{itemize}
\item Tras el periodo de inverstigación, se ha seleccionado un algoritmo y se ha implementado en Matlab. Este código es plenamente funcional y se encuentra en el apéndice \ref{ap:algoritmo}.
\item Se ha montado un circuito analógico que funciona como etapa de entrada, amplificando la señal musical y adecuándola para su correcta interpretación por parte de la FPGA.
\item Aunque no se tiene una versión completa de la implementación VHDL del algoritmo elegido, si que se han diseñado varias partes para poder estimar el funcionamiento final del sistema completo así como su arquitectura.
\end{itemize}

Dado que el proceso de realización del trabajo ha sido de un año natural, se han empleado muchas horas en cada una de las diferentes partes del mismo. Para comprender la magnitud del presente proyecto, se adjunta un desglose aproximado de las horas de trabajo en siguiente tabla: 
\begin{table}[htb]
\centering
\begin{tabular}{l r}
%%%%%%%%%%%%%%%%%%%%%%%%%%%%%%%%%%%%%%%%%%%%%%%%%%%%%%%%%%%%%%%%%%%%%%%%%%%%%%%%%%%%%%%%%%%%%%%%%%%%%%%%
\multirow{2}{9cm}{\centering{\textbf{Algoritmo}}}                   &\multirow{2}{2cm}{\textbf{150h}}  \\
				                                                    &                                  \\
\hline
Investigación de los diferentes algoritmos                          &50h                               \\ 
Comparativa y evaluación de los mismos                              &70h                               \\
Pruebas en Matlab                                                   &30h                               \\
\hline 
\hline
\multirow{2}{9cm}{\centering{\textbf{Circuito analógico}}}          &\multirow{2}{2cm}{\textbf{20h}}   \\
				                                                    &                                  \\
\hline 
Fase de documentación y montaje del circuito elegido                &15h                               \\ 
Pruebas y medidas                                                   &5h                                \\
\hline 
\hline
\multirow{2}{9cm}{\centering{\textbf{Implementación}}}              &\multirow{2}{2cm}{\textbf{320h}}  \\
				                                                    &                                  \\
\hline
Caracterización y puesta en marcha de la entrada y salida de datos  &70h                               \\ 
Implementación del algoritmo de Matlab en VHDL usando Vivado        &100h                              \\
Depuración y pruebas                                                &150h                              \\
\hline 
\hline 
\multirow{2}{9cm}{\centering{\textbf{Otros}}}                       &\multirow{2}{2cm}{\textbf{50h}}   \\
				                                                    &                                  \\
\hline
Medidas y estimaciones sobre el prototipo final                     &10h                               \\ 
Realización de la memoria                                           &40h                               \\
\hline 
\hline
\multirow{2}{9cm}{\centering{\textbf{TOTAL}}}                       &\multirow{2}{2cm}{\textbf{540h}}  \\
				                                                    &                                  \\
\end{tabular}
%\caption{\label{tabla:tiempo}Relación de tiempo de trabajo destinado a cada ámbito.}
\end{table}

Esta aproximación se ha hecho en base a la fecha de los \emph{commit} de Github realizados asumiendo una media de 12 horas de trabajo a la semana.

\section{Consecución de los objetivos propuestos}

Una vez concluido el tiempo de trabajo se pueden hacer las siguientes afirmaciones en referencia al grado de consecución de los objetivos previamente propuestos, aunque se detallen en el apartado de conclusiones:

\begin{itemize}
\item Se ha diseñado el sistema en su totalidad, atendiendo a criterios técnicos debidamente justificados en este documento.
\item Aunque no se ha llegado a implementar el sistema en su totalidad, sí que se ha trabajado en este aspecto en profundidad, permitiendo una estimación del funcionamiento de los módulos restantes que sí que se han tenido en cuenta en el diseño.
\item El prototipo evidentemente no es plenamente funcional pero sí que ha permitido la prueba de los diferentes aspectos que se han ido implementando.
\item Debido al trabajo de investigación realizado, se han adquirido nuevas ideas relacionadas con estos estos ámbitos que han sido puestas en práctica de forma crítica y se han asentado los conocimientos relacionados adquiridos durante el grado.
\end{itemize}