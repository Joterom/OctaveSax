This project belongs to the real-time music production context. Concretely, it addresses design, development and implementation of a digital effects pedal, which main feature will be an octavizer. Such devices allow us to modify the input signal in real time obtaining a variety of different effects and modulations as the outcome.

An octavizer provides a signal which is identical (ideally) to the introduced one except that its frequencies are divided by two, adding a much more solid and deep nature to the resulting mix.

This work is carried out from zero, indeed, from the original idea to the implementation of the prototype. Every single design choice and criteria has been made thinking in its professional use, refereing to either latency, audio quality, layout and all other parameters involved.

Consequently, the outcome was an evaluation of different algorithms based on its properties and characteristcs to finally make an VHDL implementation proposal using Vivado tool. This acknowledgements have been put into practise via designing a prototype that helped me to ground such concepts and improved my comprenhension of both audio signal proccessing and algorithm implementation on FPGA.
