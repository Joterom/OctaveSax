El presente proyecto se enmarca en el ámbito de la producción musical en tiempo real. En concreto plantea el diseño, desarrollo e implementación de un pedal de efectos digital. Estos dispositivos permiten añadir en tiempo real efectos y modulaciones a ciertos instrumentos musicales, incluyendo la voz humana. Un octavador es un dispositivo electrónico que recibe un sonido y mediante un procesado de señal, entrega a su salida ese mismo sonido una octava por debajo del original y va a ser el efecto principal que se va a implementar. El proyecto plantea el problema desde cero, utilizando la literatura existente para abordar tanto el diseño como la implementación.
