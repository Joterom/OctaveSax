El presente proyecto se enmarca en el ámbito de la producción musical en tiempo real. En concreto, plantea el diseño, desarrollo e implementación de un pedal de efectos digital, cuyo efecto principal será el de octavador. Estos dispositivos permiten variar la señal entrante en tiempo real obteniendo a la salida diferentes efectos y modulaciones. 

Un octavador es un dispositivo que entrega a la salida un señal que es idéntica (idealmente) a la señal de entrada salvo que su frecuencia está dividida por 2, añadiendo un carácter mucho más sólido y profundo al sonido resultante de la mezcla de ambos. 

El proyecto plantea el proyecto desde cero, es decir, desde la idea original hasta la implementación de un prototipo. Todas las decisiones y criteros de diseño se llevan a cabo pensando en su utilización profesional, refiriéndome tanto a latencia como calidad del sonido, formato, latencia y el resto de parámetros involucrados.

En consecuencia, el resultado final ha sido una evaluación de diferentes algoritmos en base a sus diferentes propiedades y características para posteriormente realizar una propuesta de implementación en VHDL utilizando la herramienta Vivado. Estos conocimientos se han plasmado en un prototipo que me ha ayudado a afianzar estos conceptos y ha mejorado mi comprensión tanto del tratamiento de señales de audio como de la implementación de un algoritmo en FPGA.
